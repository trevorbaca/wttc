\textbf{Forces:}

\begin{itemize} \itemsep2pt
\item Stage actor
\item Alto flute
\item Oboe
\item Guitar I \& II
\item Violin
\item Cello
\end{itemize}

\textbf{Accidentals.} Accidentals govern only one note. This is true even for
successive noteheads at the same staff position. The sequence of G$\sharp$4
followed by G4 (without accidental) is to be understood as G$\sharp$4 followed
by G$\natural$4.

\textbf{Appoggiaturas.} Play runs of small-note appoggiaturas as fast as
possible starting directly on the beat; land immediately on the full-size note
shown below and sustain to the end of the duration indicated.

\textbf{Glissandi.} Although the start-pitches of glissandi should be clearly
articulated, neither the stop-pitch nor any intermediate pitches need to be
rearticulated. This is as true for single-segment portamenti in the alto flute
as for the elaborate multistage glissandi in the violin and cello. In other
words, all nonfirst pitches in glissandi may be thought of as being small
parenthesized notes (whether they are in fact notated this way or not).
Quarter-note stems are included to clarify the rhythm of some longer glissandi;
these stems are rhythmic indicators only and do not change the sound of the
glissandi produced. Play all glissandi as smooth and continuous, rather than
angular or disjointed.

\textbf{Flat glissandi.} Flat glissandi are sometimes used as a typographical
variant of ties.

\textbf{``Terminate abruptly.''} The symbol of a capital T turned on its side
is used to mean that the sound so-marked is to terminated abruptly (ripped or
``gerisst''), rather than being allowed to fade away of its own accord. To
accomplish this, the flute should bring the tongue to the back of the teeth (or
alveolar ridge of the gums) to stop the air; the oboe should touch the tongue
to the reed to stop the air; and the violin and cello use the bow to choke the
sound at the end of its duration, termating such notes stop-on-string. The
symbol is written for the guitars only over string-scraping gestures; terminate
these in similar fashion to the violin and cello, using the scraper to suddenly
choke the sound, marking the end of the note stop-on-string. In all cases, the
sound ideal is of a decidedly `square' or `rectangular' sound that treats the
termination of the sound a distinct event.

\textbf{Winds.} Trills without secondary pitches are bisbigliandi, or color
trills. Pick a secondary fingering that nearly matches the pitch of the primary
note. If no color fingering is available, play the secondary pitch a minor
second higher than the primary pitch.

\textbf{Alto flute.} The alto flute sounds a perfect fourth lower than written.
Whistle tones are shown as runs of small note-heads written with neither stems
nor dynamics; finger these over the pitch-class of the first note in such
passages. Play passages marked ``covered'' (or ``cov.'') by covering the
opening of the flute with the lips, sounding a perfect fourth plus a major
seventh lower than written; such passages sometimes (though not always) are
notated as dyads to show the sounding pitch. Airtone is marked as such,
sometimes abbreviated ``air.''

\textbf{Guitars.} No scordatura; the guitars sound an octave lower than
written. In general, the sound ideal for the guitars is as resonant and
luminous as possible. Although many lone notes are equipped with laissez-vibrer
ties, essentially all passages in the guitars can be played with glowing
resonance. This is especially true prior to fermatas and other silences, where
after-echo of the guitars is especially to be encouraged.

\textbf{Doits and falls.} Doits and falls are usually written at an
interval of a fourth; follow the up- or down-direction of doits and falls,
but stretch or compress the distance of these figures \textit{ad lib}.

\textbf{Lateral scraping up the string.} Play passages written on a 1-line
staff by scraping laterally up the wire wrapping of the low E-string with a
credit card, plastic pick, small coin or other item. The goal is a rhythmically
precise burst of ``ribbed'' or ``serrated'' or ``guiro-like'' white noise.
(These passages frequently articulate a type of rhythmic canon between the two
guitars.) Follow the up-bow and down-down indications given.

\textbf{Metal-screw ``rasgueado.''} At the end of each of the four ``Stills''
in the piece a note is marked ``with screw.'' The indication here means to
finger the pitch indicated while bowing the low E string with a long,
lightweight metal machinist's screw, available at most hardware stores. Make
sure the screw is at least 8 -- 12 inches long, which seems to help with the
movement of the screw over the wraping of the string. When bowing, draw the
screw across the string at a spot only a fret or two away from the finger of
the left hand: trial and error will reveal at string contact point that elicits
a cascade of glowing metallic ``flecks'' in a dancing spectrum of sound rooted
at the pitch fingered by the left hand. Jürgen Ruck played a first version of
this sound in my piece \textit{L'archipel du corps} in 2011. Nico Couck
developed the version of the sound called for here in \textit{Spiel der
Dornen}, premiered at Darmstadt in 2016.

\textbf{Strings.} No scordatura; the violin and cello sound as written.
\textbf{LH damping.} Passages marked with a damp (``crosshairs'') symbol should
be played with the left hand damping the string at the position indicated:
lightly lay three fingers on the string to produce a beautiful grey sound with
perceptible (but muted) sense of pitch. \textbf{XFB.} Passages marked ``XFB''
(``extremely fast bow'') should be played with a fast, extremly light,
desynchronized type of tremolo flautando: use generous amounts of bow and
change the bow irregularly (while noting that the technique is decidedly less
hectic than it might first appear; the bow only skims the surface of the string
throughout: do not play ``into'' the string at all). Most XFB passages seem to
be helped by playing somewhat tasto on the string. The aural result of the
technique is a ``fluorescent'' type of flautando that brings out the middle
partials of the string's sound. \textbf{Rimbalzandi.} Triple-staccati indicate
rimbalzandi; these do not mean that the bow should bounce exactly three times;
rather, play as many bounces of the bow as allowed at the current tempo.
\textbf{Bowing on the wood of the instrument.} Play passages notated on the
1-line staff directly on the wood of the instrument, with the goal being as
lucious and beautiful and ``snowlike'' a sound as possible; the exact location
at which the bow is drawn is left to the performer; directly on the bridge, on
the side of the instrument, on a wooden mute, are all possible, though bowing
the tailpiece should be avoided for reasons of the pitch that tailpiece bowing
adds to the sound. \textbf{String contact point (SCP) transitions.} Transitions
between ponticello (P), ordinario (O) and tasto (T) string contact points are
shown with arrows; P1, P2, P3, P4 indicate string contact points progressively
closer to the bridge (and brighter and more acidic in timbre); T1, T2, T3, T4
indicate string contact points progressively closer to the nut (and mellower
and smoother in timbre). \textbf{Natural harmonics and half-harmonics.} White
diamond noteheads indicate natural harmonics in the usual way; black diamond
noteheads indicate half harmonic pressure. \textbf{Artificial harmonics.}
Artificial harmonics at the fourth sound two octaves higher than the
fundamental, in the usual way; artificial harmonics at the major third sound a
major third two octaves higher than the fundamental; artificial harmonics at
the minor third sound a perfect fifth two octaves higher than the fundamental;
and artificial harmonics at the perfect fifth sound a twelfth higher than the
fundamental. \textbf{Full-bow strokes.} Up-bow and down-bow symbols equipped
with dangling tails indicate complete bow strokes in the direction given. The
symbols provide for very fast movements of the bow, usually played half col
legno tratto. \textbf{Half col legno tratto.} Play passages marked ``1/2 clt''
with the bow rotated to allow both hair and wood to travel across the string.
The goal is to introduce a healthy amount of whisking into the sound,
especially when combined with full up-bow and full down-bow strokes.
\textbf{Circle bowing.} Circling bowing is indicated with an arrowed circle,
with exactly one full circle of the bow for each such symbol. Note that because
the notes in such passages are many times rhythmed differently, the speed of
each circle may vary quite a bit: very fast circles are required over sixteenth
notes while much slower circles are required over half notes and whole notes.
Aim at a luminous color in circle-bowed passages; the sound ideal is a
strikingly animated type of glow, which seems to be helped by making wide
circles that allow for generous travel of the bow during each stroke. (Gritty,
granulated, tight circles are to be avoided.) \textbf{Bariolage.} Adjust
slurring in the passages of bariolage as necessary. \textbf{Pizzicati.} Though
conventional (RH) pizzicati are usually indicated in the normal way
(``pizz.''), it is sometimes the case that a cross is also used to indicate RH
pizzicati. That is, there are no LH pizzicati in the piece; play all notes
marked with a cross as conventional RH pizzicati, not LH pizzicati. (Notation
to be revised.) \textbf{Two-finger pizzicato.} Play passages marked ``2f.
pizz.'' as a type of rapid tremolo pizzicato, with the first two fingers
plucking the string in rapid alternation; such passages are almost always
scored for violin and cello together.
