\textbf{Strings.} No scordatura; the violin and cello sound as written.
\textbf{LH damping.} Passages marked with a damp (``crosshairs'') symbol should
be played with the left hand damping the string at the position indicated:
lightly lay three fingers on the string to produce a beautiful grey sound with
perceptible (but muted) sense of pitch. \textbf{XFB.} Passages marked ``XFB''
(``extremely fast bow'') should be played with a fast, extremly light,
desynchronized type of tremolo flautando: use generous amounts of bow and
change the bow irregularly (while noting that the technique is decidedly less
hectic than it might first appear; the bow only skims the surface of the string
throughout: do not play ``into'' the string at all). Most XFB passages seem to
be helped by playing somewhat tasto on the string. The aural result of the
technique is a ``fluorescent'' type of flautando that brings out the middle
partials of the string's sound. \textbf{Rimbalzandi.} Triple-staccati indicate
rimbalzandi; these do not mean that the bow should bounce exactly three times;
rather, play as many bounces of the bow as allowed at the current tempo.
\textbf{Bowing on the wood of the instrument.} Play passages notated on the
1-line staff directly on the wood of the instrument, with the goal being as
lucious and beautiful and ``snowlike'' a sound as possible; the exact location
at which the bow is drawn is left to the performer; directly on the bridge, on
the side of the instrument, on a wooden mute, are all possible, though bowing
the tailpiece should be avoided for reasons of the pitch that tailpiece bowing
adds to the sound. \textbf{String contact point (SCP) transitions.} Transitions
between ponticello (P), ordinario (O) and tasto (T) string contact points are
shown with arrows; P1, P2, P3, P4 indicate string contact points progressively
closer to the bridge (and brighter and more acidic in timbre); T1, T2, T3, T4
indicate string contact points progressively closer to the nut (and mellower
and smoother in timbre). \textbf{Natural harmonics and half-harmonics.} White
diamond noteheads indicate natural harmonics in the usual way; black diamond
noteheads indicate half harmonic pressure. \textbf{Artificial harmonics.}
Artificial harmonics at the fourth sound two octaves higher than the
fundamental, in the usual way; artificial harmonics at the major third sound a
major third two octaves higher than the fundamental; artificial harmonics at
the minor third sound a perfect fifth two octaves higher than the fundamental;
and artificial harmonics at the perfect fifth sound a twelfth higher than the
fundamental. \textbf{Full-bow strokes.} Up-bow and down-bow symbols equipped
with dangling tails indicate complete bow strokes in the direction given. The
symbols provide for very fast movements of the bow, usually played half col
legno tratto. \textbf{Half col legno tratto.} Play passages marked ``1/2 clt''
with the bow rotated to allow both hair and wood to travel across the string.
The goal is to introduce a healthy amount of whisking into the sound,
especially when combined with full up-bow and full down-bow strokes.
\textbf{Circle bowing.} Circling bowing is indicated with an arrowed circle,
with exactly one full circle of the bow for each such symbol. Note that because
the notes in such passages are many times rhythmed differently, the speed of
each circle may vary quite a bit: very fast circles are required over sixteenth
notes while much slower circles are required over half notes and whole notes.
Aim at a luminous color in circle-bowed passages; the sound ideal is a
strikingly animated type of glow, which seems to be helped by making wide
circles that allow for generous travel of the bow during each stroke. (Gritty,
granulated, tight circles are to be avoided.) \textbf{Bariolage.} Adjust
slurring in the passages of bariolage as necessary. \textbf{Pizzicati.} Though
conventional (RH) pizzicati are usually indicated in the normal way
(``pizz.''), it is sometimes the case that a cross is also used to indicate RH
pizzicati. That is, there are no LH pizzicati in the piece; play all notes
marked with a cross as conventional RH pizzicati, not LH pizzicati. (Notation
to be revised.) \textbf{Two-finger pizzicato.} Play passages marked ``2f.
pizz.'' as a type of rapid tremolo pizzicato, with the first two fingers
plucking the string in rapid alternation; such passages are almost always
scored for violin and cello together.
